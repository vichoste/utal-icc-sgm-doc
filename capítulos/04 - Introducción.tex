\section{Contexto del proyecto}
La facultad de Ingeniería de la Universidad de Talca, ubicada en el campus Curicó, ofrece a los estudiantes la carrera de Ingeniería Civil en Computación, la que prepara a un futuro profesional con una formación en Ciencias Básicas e Ingeinería, así como en una variedad de disciplinas de la computación, tales como Ingeinería de Software, Análisis de Datos, entre otros \cite{icc}.
Esta carrera tiene una duración esperada de 11 semestres por estudiante, de acuerdo a la malla curricular en vigencia en el año 2023: cada alumno debe realizar los cursos presentes en la malla curricular, los cuales se dividen en cursos obligatorios y electivos. Los cursos obligatorios son aquellos que son necesarios para que el alumno pueda obtener el título, mientras que los electivos son aquellos que el alumno puede elegir para complementar su formación académica; se suman también dos prácticas profesionales, las cuales son obligatorias y se realizan en empresas externas a la universidad o bien recomendadas por la misma \cite{icc2019}.

Dentro de los cursos presentes dentro de la malla curricular, se encuentran los módulos de Formulación de Proyecto de Titulación y Proyecto de Titulación -en los dos últimos semestres-, los cuales consisten en la realización de un proyecto de software que resuelva un problema en particular, el cual debe ser propuesto por el alumno, o bien, por un profesor de la carrera.
El alumno debe utilizar las herramientas y conocimientos adquiridos durante su formación académica para integrarlos en el desarrollo del proyecto, así como comunicar de manera efectiva los resultados obtenidos, modelar el problema y su solución, redactar un documento que describa el proceso, y finalmente, defender el proyecto ante un comité de profesores de la carrera \cite{fpt2019, pt2019}.

\section{Definición del problema}
Al momento de realizar el proyecto de titulación, el alumno debe definir un problema a resolver, el cual debe ser propuesto por el alumno, o bien, por un profesor de la carrera, el cual guía al alumno durante el desarrollo del proyecto.
El problema debe ser lo suficientemente complejo como para que el alumno pueda aplicar los conocimientos adquiridos durante su formación académica, pero no tan complejo como para que el alumno no pueda resolverlo en el tiempo disponible para la realización del proyecto.
Además, el problema debe ser lo suficientemente amplio y específico como para que el alumno pueda realizar un proyecto de software que lo resuelva, pero no tan amplio como para que el alumno no pueda resolverlo en el tiempo disponible para la realización del proyecto.

No existe una regulación formal con respecto al proceso de definición del problema, así como los alcances del proyecto; el alumno podría definir un problema que no sea lo suficientemente complejo, o bien, que no sea lo suficientemente amplio, o bien, que no sea lo suficientemente específico, lo que podría llevar a que el alumno no pueda resolver el problema en el tiempo disponible para la realización del proyecto, o bien, que el proyecto no cumpla con los objetivos propuestos.
También se puede dar el caso de que el profesor guía no tenga conocimientos suficientes para poder guiar al alumno durante el desarrollo del proyecto o bien, en el caso contrario -es decir, que el profesor tenga muchos conocimientos sobre el tema del proyecto- que le exija más de lo que el alumno puede entregar, lo que podría llevar a que el alumno no pueda resolver el problema en el tiempo disponible para la realización del proyecto, o bien, que el proyecto no cumpla con los objetivos propuestos.



\section{Propuesta de solución}

\section{Objetivos}
\paragraph{Objetivo general}
\begin{itemize}
	\item Test
\end{itemize}
\paragraph{Objetivos específicos}
\begin{itemize}
	\item Test
\end{itemize}

\section{Alcances del proyecto}
\begin{itemize}
	\item Test
\end{itemize}